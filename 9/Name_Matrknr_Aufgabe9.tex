\documentclass[12pt,a4paper]{report}
\usepackage{amssymb,amsthm,amsmath,amscd}
\usepackage{latexsym}
\usepackage{enumerate}
\usepackage[german]{babel}
\usepackage{verbatim}
\usepackage[hyphens]{url}
\usepackage{hyperref}
\usepackage[utf8]{inputenc}
\usepackage{pdfpages}
\usepackage{graphicx}
\usepackage{csquotes}
%\usepackage[landscape]{geometry}
\begin{document}
\begin{titlepage}
	\begin{center}

		\vspace*{1.0cm}
		\huge
		\textsc{\bf{PS Algorithmen für verteilte Systeme}}

		\vspace*{4.0cm}
		\textsc{
			\normalsize{eingereicht von} \\[0.5\baselineskip]
			{\large Baumgartner Dominik, Dafir Samy}
		}

		\vspace*{3.0cm}
		\textsc{
			\normalsize{Gruppe  1(16:00)}
		}

	\end{center}
\end{titlepage}

\section*{Aufgabe 16}

\subsection*{Implementierung:}
Für diesen Teil wurde der Rumor-Spreading Algorithmus (Model I) nach Elsässer und Sauerwald implementiert und empirisch hinsichtlich Laufzeit (Runden) und versendeten Nachrichten ausgewertet. Der implementierte Algorithmus wurde auf einem 
Zufallsgraphen $G_{n,p}$ mit n = 10000 und p = 0.005 getestet.
Der Publikation \"The Power of Memory in Randomized Broadcasting\" zufolge kann dieser Algorithmus alle Knoten in einem
Graphen innerhalb $O(ln(n))$ Runden und mit $O(n * log(log(n)))$ Nachrichten informieren.\\
Empirische Untersuchungen lassen auf die obigen Laufzeiten schließen.
Dazu wurde die Informationsverteilung auf 300 zufällig erstellten Graphen mit obigen Eigenschaften  ausgeführt.
Um eine zufällige Übereinstimmung der Laufzeit mit Gaphen der Größe 10000 auszuschließen wurde der Algorithmus
zusätzlich auf 40 Zufallsgraphen der Größe 100000 angewandt. Auch dies ergab eine Übereinstimmung mit der asymptotischen
Laufzeit des Algorithmus. Im folgenden wurden nun 2 Graphen erstellt, die die Laufzeiten und Nachrichtenanzahlen für 10000, sowie 100000 Knoten zeigen.




\end{document}





















